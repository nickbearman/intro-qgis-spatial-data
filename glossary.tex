%line spacing?

%sudo /usr/bin/tlmgr install siunitx to install package

\documentclass[a4paper,10pt]{article}
   
\usepackage{geometry}
 \geometry{
 a4paper,
 total={170mm,257mm},
 left=15mm,
 right=15mm,
 top=15mm,
 bottom=10mm,
 }       
\usepackage[utf8]{inputenc}
\usepackage[english]{babel}
\usepackage{graphicx}
\usepackage{wrapfig}
\usepackage{siunitx} %for degree symbol
\usepackage{multicol}
\usepackage[ddmmyyyy,hhmmss]{datetime}

\setlength{\parskip}{0.5em} %gap between paragraphs

\setlength{\columnsep}{5mm} %column separation

\begin{document}

\pagenumbering{gobble} %disable page numbering, https://tex.stackexchange.com/questions/7355/how-to-suppress-page-number

\begin{center}

{\huge GIS Glossary}

% \textit{Terms in italics are defined in the glossary}

\end{center}

\begin{multicols}{2}

% \textbf{Acknowledgements} Information required on any map, including copyright (e.g. for OpenStreetMap or Ordnance Survey Open Data) or data sources 

\textbf{ArcCatalog} Part of the \textit{ArcGIS} package, primarily used for managing spatial files such as \textit{shapefiles} % and \textit{personal geodatabases}

\textbf{ArcGIS} A commercial GIS software created by ESRI, consisting of \textit{ArcMap}, \textit{ArcCatalog} and \textit{ArcScene} 

\textbf{ArcMap} Part of \textit{ArcGIS}, the main program for creating and editing spatial data and maps

\textbf{ArcScene} Part of \textit{ArcGIS}, used for 3D data 

\textbf{Attribute table} The table of additional information associated with each \textit{shapefile} (e.g. country names); access by right-clicking on the layer and selecting Open Attribute Table 

\textbf{BNG} (British National Grid) A \textit{coordinate system} used to represent locations in Great Britain, consisting of \textit{eastings} and \textit{northings}, e.g. 603125, 112589 (see also \textit{UTM} and \textit{WGS1984}) 

\textbf{Categorical} A variable that has a series of values with no inherent order, e.g. country names, also known as \textit{nominal} (see also \textit{variable type}, \textit{quantitative}) 

\textbf{Choropleth} A type of mapping where different colours are used to represent difference values; can use \textit{categorical} and \textit{ordinal} data

\textbf{Classes} The groups data are put into for a \textit{choropleth} map

\textbf{Classification} How the data are classified into different \textit{classes} for a choropleth map (see also \textit{jenks}, \textit{equal count}, \textit{equal interval} and \textit{standard deviation}

\textbf{Coordinates} The numbers representing a specific location, usually presented in pairs (see also \textit{latitude}, \textit{longitude}, \textit{WGS1984}, \textit{BNG} and \textit{UTM})

\textbf{Coordinate system} The type of coordinates that are used to represent a specific location (see also \textit{WGS1984}, \textit{BNG} and \textit{projection})

\textbf{Correlation} A measure of how much two variables are related, measured using a \textit{R\textsuperscript{2}} value 

\textbf{CSV} (Comma separated values) A standard format of \textit{tabular data}, can be opened in Excel  

\textbf{CSVT} An optional file for use with \textit{CSV} files which specifies the \textit{variable type} of each column % in the \textit{CSV} file 

\textbf{Data frame} (ArcMap) A section of the map in Layout View containing specific layers of spatial data

\textbf{Data type} How data is stored within the \textit{Attribute table}, can be \textit{integer} (whole numbers), \textit{real} (decimal numbers) and \textit{string} (text) 

\textbf{DEM} Digital Elevation Model, a \textit{raster} representation of the height of the earth's surface

\textbf{Eastings} A \textit{coordinate} that specifies the distance east, in meters, from the coordinates 0,0 south-west of the Isles of Scilly (see also \textit{BNG} and \textit{northings})

\textbf{Equal count} (Quantile) \textit{Classification} method where data are split into a number of groups by putting the same number of data items into each group, also known as \textit{quantile}, see also \textit{classification}

\textbf{Equal interval} \textit{Classification} method where data are split into \textit{classes} that are evenly distributed, e.g. 0-20\%, 20-40\%, etc., see also \textit{classification} 

\textbf{Feature class} One layer within a \textit{personal geodatabase}; can contain one of \textit{points, lines} and \textit{polygons}

\textbf{Field calculator} Used to calculate new values (e.g. differences) from existing values for all rows in a vector layer, accessed from the \textit{Attribute table} 

% \textbf{Fisher} \textit{Classification} method very similar to \textit{Jenks} 

\textbf{Geodatabase} See \textit{personal geodatabase}

\textbf{Geographic Information Science} (GIS) The development of the tools, software and processes used in \textit{Geographic Information Systems} 

\textbf{Geographic Information Systems} (GIS) Using spatial data to answer questions about our world (see also \textit{Geographic Information Science})

\textbf{GeoJSON} Vector spatial data file, consisting of \textit{points}, \textit{lines} and \textit{polygons}; all saved in one file

\textbf{GPS} (Global Positioning System) a series of 24 satellites in orbit around the earth which allow a GPS device to locate itself, with an accuracy of 1m to 10m

\textbf{Inset Map} A small map included on the main map to aid orientation, e.g. a map of Ghana might include an \textit{inset map} of Africa to show where Ghana is

\textbf{Integer} A whole number used to represent data, can be used in a \textit{choropleth} map (see also \textit{data type}) 

\textbf{Jenks} (natural breaks) \textit{Classification} method based on the Jenks algorithm which groups similar data values together, also known as \textit{natural breaks}, see also \textit{classification}

\textbf{Joining} The process of linking attribute information to spatial data, often used so the information can be shown on a \textit{choropleth} map 

\textbf{Latitude} A \textit{coordinate} that specifies the distance north or south, ranging from \ang{0} at the Equator to \ang{90} (North or South) at the poles (see also \textit{WGS1984} and \textit {longitude})

\textbf{Layers} When you add data into a GIS each different file appears as a different \textit{layer}; this allows different datasets to be overlaid on one another (see also \textit{Table of contents} and \textit{Layers window})

\textbf{Layers window} (QGIS) Panel on the left hand side of QGIS, showing the different GIS layers in your map; the order of the layers can be changed (known as the \textit{Table of contents} in \textit{ArcMap})

\textbf{Legend} An important part of any map, showing what the symbols or colours used on the map represent 

\textbf{Lines} Used in \textit{vector} data sets to indicate a linear feature, such as rivers, roads or railways; is a series of \textit{points} joined together with lines

\textbf{Longitude} A \textit{coordinate} that specifies the distance east or west, ranging from \ang{0} at the Prime Meridian to \ang{180} (East or West) (see also \textit{WGS1984} and \textit{latitude})

\textbf{MapInfo} A commercial GIS software, created by Pitney Bowes 

\textbf{MXD project file} (.mxd) (ArcMap) A project file for \textit{ArcMap} which contains links to all the data files e.g. \textit{shapefiles} or \textit{geodatabases}) and information on how they are symbolised; the \textit{MXD} file does not contain the data itself (see also \textit{QGIS project file})

\textbf{Nominal} A variable that has a series of values with no inherent order, e.g. country names, also known as \textit{categorical} (see also \textit{variable type}, \textit{ordinal} and \textit{quantitative})

\textbf{North arrow} Used to show the direction of North on a map, used to aid orientation (see also \textit{inset map})

\textbf{Northings} A coordinate that specifies the distance north, in meters, from the \textit{coordinates} 0,0 south-west of the Isles of Scilly (see also \textit{BNG} and \textit{eastings})

\textbf{Ordinal} Similar to a categorical variable, but with a clear order, e.g. high priority, medium priority, and low priority (see also \textit{variable type}, \textit{quantitative}) 

\textbf{Personal geodatabase} A type vector of spatial data file, consisting of one or more \textit{feature classes}; can only be used in \textit{ArcGIS} (see also \textit{feature class})

\textbf{Pixel} An individual unit in a \textit{raster} data set, the size of the \textit{resolution} squared (i.e. for a 100m resolution \textit{raster} data set, each \textit{pixel} would be 100m x 100m, covering 10,000 square meters (or 1 hectare) of land)

\textbf{Points} A \textit{vector} data type used to indicate a specific location, such as sample collection points, bird nest sites, towns or cities

\textbf{Polygons} A \textit{vector} data type used to indicate areas, e.g. land parcels, counties and fields; is a series of \textit{points} joined with \textit{lines} and closed to indicate an area

\textbf{Print composer} The tool in QGIS used to design maps and add a \textit{legend}, \textit{scale bar}, \textit{north arrow} and any required acknowledgements or copyright  

\textbf{Projection} The way the sphere shaped earth is distorted to fit on a flat piece of paper (see also \textit{WGS1984}, \textit{BNG} and \textit{coordinate system})

\textbf{QGIS} \textit{(previously Quantum GIS)} An open source GIS created as broadly similar to \textit{ArcMap}  which is free for anyone to download, use and improve

\textbf{QGIS project file} (.qgs) (QGIS) A project file for \textit{QGIS} which contains links to all the data files (such as \textit{shapefiles} and/or \textit{GeoJSON} files) and information on how they are symbolised; the \textit{project file} does not contain the data itself (see also \textit{MXD file})

\textbf{Quantile} (equal count) \textit{Classification} method where data are split into a number of groups by putting the same number of data items into each group, also known as \textit{equal count}, see also \textit{classification}

\textbf{Quantitative} A numeric variable with an inherent order, e.g. GDP per capita, (see also \textit{variable type})

\textbf{R\textsuperscript{2}} The \textit{correlation} coefficient of two different data sets, a value of 1 is a strong positive \textit{correlation}, -1 is a strong negative \textit{correlation}

\textbf{Raster} A type of spatial data used with GIS, consisting of a regular grid of points spaced at a set distance (the \textit{resolution}); often used to represent heights (DEM) or temperature data (see also \textit{vector})

\textbf{Raster calculator} Used with \textit{raster} data to calculate differences (subtract) or calculate other indices (e.g. NDVI)

\textbf{Real} A decimal number used to represent data, can be used in a \textit{choropleth} map (see also \textit{data type}) 

\textbf{Resolution} The size of each \textit{pixel} in a \textit{raster} data set (e.g. 100 meters, 1km, 100km) (see also \textit{pixel}) 

\textbf{Sat-nav} A navigation system in cars, which uses \textit{GPS} to direct the driver to their destination

\textbf{Scale} The ratio of units of distance on the map to units of distance in the real world; for example 1:25,000 means that 1cm on the map represents 25,000cm (or 250m) in the real world; usually shown on a \textit{scale bar}

\textbf{Scale bar} Used to show the \textit{scale} of a map

\textbf{Shapefile} A type vector of spatial data file, consisting of one of \textit{points}, \textit{lines} or \textit{polygons}; represented in \textit{GIS} as one file but in fact consisting of multiple files (between 4 and 6 files, with extensions of .shp, .dbf, .shx and  .prj)

\textbf{Standard deviation} \textit{Classification} method based on standard deviation and mean of the data set  

\textbf{String} A piece of text (e.g. a name) used to represent data, cannot be used in a \textit{choropleth} map (see also \textit{data type}, \textit{real} and \textit{integer} 

\textbf{Style} (QGIS) / \textbf{Symbology} (ArcMap) The options to choose the colours and/or symbols to represent data on the map; accessed through right-clicking on the layer and selecting properties and navigating to the Style tab)

\textbf{Table of contents} (ArcMap) Panel on the left hand side of \textit{ArcMap}, showing the different \textit{GIS} \textit{layers} in your map; the order of the layers can be changed (known as the \textit{Layers window} in QGIS)

\textbf{Tabular data} Data laid out in rows and columns, as used in Excel (see also \textit{CSV})

\textbf{UTM} (Universal Transverse Mercator) A type of \textit{coordinate system} used to represent any location in the world, consisting of a series of zones and a set of \textit{coordinates} for each zone, in meters (see also \textit{BNG} and \textit{WGS1984}) 

\textbf{Variable type} Information on the type of information within a variable, can be \textit{categorical}, \textit{ordinal} or \textit{nominal} 

\textbf{Vector} A type of spatial data used with \textit{GIS}, consisting of \textit{points}, \textit{lines} and \textit{polygons} (see also \textit{raster})

\textbf{Vertex (vertices)} Name for each of the points that connect the \textit{line} segments of a \textit{line} or \textit{polygon} \textit{shapefile}

\textbf{WGS1984} A \textit{coordinate system} used to represent any location in the world, consisting of \textit{latitude} and \textit{longitude} e.g. 51.0426 N, 1.3772 E or \ang{51} 2’ 33.53’’ N, \ang{1} 22’ 38.23’’ E (see also \textit{BNG} and \textit{UTM}) 

\end{multicols}

\begin{center}

{\footnotesize \textit{This glossary was last updated on {\today} by Dr. Nick Bearman (nick@geospatialtrainingsolutions.co.uk) and is written in LaTeX. This work is licensed under the Creative Commons Attribution-ShareAlike 4.0 International License, http://creativecommons.org/ licenses/by-sa/4.0/deed.en. The latest version of the PDF is at https://github.com/nickbearman/intro-qgis-spatial-data.}}

\end{center}

\end{document}